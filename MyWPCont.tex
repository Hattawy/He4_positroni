\documentclass[times, twoside]{PosWhiPap}
\usepackage{blindtext}

% Please give the surname of the lead author for the running footer
\leadauthor{S.~Fucini {\it et al.}} 

\begin{document}

\title{DVCS off He nuclei with positron beams}
\shorttitle{Positrons on He targets}

% Use letters for affiliations, numbers to show equal authorship (if applicable) and to indicate the corresponding author
%\author[1,\Letter]{Ricardo Henriques}
\author[]
{S.~Fucini??, M.~Hattawy??, M.~Rinaldi??, S.~Scopetta???  }

%\affil[1]{Quantitative Imaging and Nanobiophysics Group, MRC Laboratory for Molecular Cell Biology and Department of Cell and Developmental Biology, University College London, Gower Street, London, WC1E 6BT, United Kingdom}

\maketitle

%TC:break Abstract
%the command above serves to have a word count for the abstract
\begin{abstract}
%\blindtext
The relevance of using (polarized) positron beams in DVCS
off $^4$He and $^3$He is addressed.
The way the so-called $d-$term could be extracted from the real
part of the relevant Compton form factor, using as an example
coherent DVCS on $^4$He, is summarized.
The importance and novelty of such a measurement is stressesed.
The role of $^3$He targets, of incoherent (tagged) DVCS processes,
and the unique possibility offered by postron beams for the investigation of  Compton form factors of higher twist, are also  briefly addressed.
\end {abstract}
%TC:break main
%the command above serves to have a word count for the abstract

%\begin{keywords}
%bla | bla | bla | bla
%\end{keywords}

%\begin{corrauthor}
%\texttt{r.henriques{@}ucl.ac.uk}
%r.henriques\at ucl.ac.uk
%\end{corrauthor}

%\section*{Introduction}

In the last few years, there has been a
growing interest on nuclear DVCS, thanks essentially to
two reasons
i)) the possibility to shed light on the EMC
effect, i.e., the nuclear modifications of the nucleon parton structure
\cite{Dupre:2015jha,Cloet:2019mql}
and ii) the possibility to distinguish
coherent and incoherent channels
of nuclear DVCS, demonstrated recently by the CLAS collaboration at JLab using a $^4$He target \cite{Hattawy:2017woc,Hattawy:2018liu}.

To fix the ideas on how positron beams could help in this field, 
let us think to coherent DVCS off $^4$He.
%As observed for the nucleon target before in this White Paper,
%positrons allow a precise extraction of the Re CFF.
We recall that
$^4$He has only one chiral even 
Compton Form Factor (CFF) at leading twist.

In the EG6 experiment the crucial measured
observable was the single-spin asymmetry
$A_{LU}$, which can be 
extracted from the reaction yields for the two electron
helicities ($N^{\pm}$):
\begin{equation}
A_{LU} = \frac{1}{P_{B}} \frac{N^{+} - N^{-}}{N^{+} + N^{-} },
\end{equation}
where $P_{B}$ is the degree of longitudinal polarization of the incident 
electron beam.
 In EG6 kinematics, 
the cross section of real photon 
electroproduction is dominated by the BH contribution, while the DVCS 
contribution is very small. However, the DVCS contribution is
enhanced in the observables sensitive to the interference term, {\it e.g.} 
$A_{LU}$, which depends on the
azimuthal angle $\phi$ between the $(e,e^\prime)$ and 
$(\gamma^*,^4$He$^\prime)$ planes.
The asymmetry $A_{LU}$ 
for a spin-zero target can be approximated at leading-twist as
\begin{equation}
A_{LU}(\phi) = 
\frac{\alpha_{0}(\phi) \, \Im m(\mathcal{H}_{A})} 
{den(\phi)} \, ,
\end{equation}
\begin{eqnarray}
den(\phi) & = & 
\alpha_{1}(\phi) + \alpha_{2}(\phi) \, \Re e(\mathcal{H}_{A}) 
\nonumber
\\
& + & \alpha_{3}(\phi) \, 
\big( \Re e(\mathcal{H}_{A})^{2} + \Im m(\mathcal{H}_{A})^{2} \big)\, .
\label{boh}
\end{eqnarray}
The kinematic factors $\alpha_i$ are known
(see, e.g., Ref. \cite{Belitsky:2001ns,Belitsky:2008bz}). Using the different $\sin(\phi)$ and $\cos(\phi)$ contributions, 
in the experimental analysis,
both the real and imaginary part 
of the so-called Compton Form Factor $\mathcal{H}_{A}$,
$\Re e(\mathcal{H}_{A})$ and
$\Im m(\mathcal{H}_{A})$, respectively,
have been extracted 
by fitting the $A_{LU}(\phi)$ distribution.

Theoretical calculations from Ref. \cite{Fucini:2018gso} are shown together with the data of Ref. \cite{Hattawy:2017woc} in Figs. \ref{uno} and \ref{due}.
Big statistical errors are seen everywhere but they are bigger for 
$\Re e(\mathcal{H}_{A})$ than for
$\Im m(\mathcal{H}_{A})$,
due to the small coefficient $\alpha_2$.

Realistic theoretical calculations are possible
for light nuclei and could help to unveil an exotic 
behavior of the real part of $\mathcal{H}_{A}$.
Forth-coming data form
JLab 12 with electrons, using also the detector system
developed by the ALERT run-group \cite{Armstrong:2017wfw}, 
will obtain smaller errors;
anyway $\Re e(\mathcal{H}_{A})$ will be always less precise than
$\Im m(\mathcal{H}_{A})$, intrinsically, due to that small coefficient.
The knowledge of $\Re e(\mathcal{H}_{A})$ would be instead crucial.
Positrons would guarantee it, because combining data
for asymmetries measured using electrons and positrons
the role of $\Re e \mathcal{H}_{A}$ would be directly accessed.
Let us recall how it is possible.

One should notice that, between the quantities appearing in the above
equations and the
cross sections defining the generic photo-$e^\pm$production cross section
in the following schematic general expression, previously given in this White Paper,
\begin{eqnarray}
\sigma^e_{\lambda 0}  & = & \sigma_{BH} + \sigma_{DVCS} + \lambda \tilde \sigma_{DVCS} 
\nonumber
\\
& + & e\sigma_{INT} + e \lambda \tilde \sigma_{INT} \, ,
\label{gen}
\end{eqnarray}

the following relations hold:
\begin{eqnarray}
\sigma_{BH} & \propto & \alpha_1(\phi)\, ,
\nonumber \\
\sigma_{DVCS} & \propto &   \alpha_{3}(\phi) 
\big( \Re e(\mathcal{H}_{A})^{2} + \Im m(\mathcal{H}_{A})^{2} \big) \, ,
\nonumber \\
\sigma_{INT} & \propto & \alpha_{2}(\phi) \, \Re e(\mathcal{H}_{A}) \, ,
\nonumber \\
\tilde \sigma_{INT} & \propto & \alpha_{0}(\phi) \, \Im m(\mathcal{H}_{A}) \, ,
\end{eqnarray}

while $\tilde \sigma_{DVCS} $ is proportional to a term kinematically suppressed
at JLab kinematics, dependent on higher twist CFFs. 

From a combined analysis of data taken with polarized electrons or positrons,
one could access all the five cross sections in  \eqref{gen}.
In particular, using only unpolarized electrons and positrons, 
$\Re e(\mathcal{H}_{A})$ would be directly accessed.




In particular, let us briefly 
analyze why the knowledge of $\Re e\mathcal{H}_{A}$ would be very
important for nuclei.
From a theoretical point of view, one can write,
for the quantities $\Re e(\mathcal{H}_{A})$ 
and $\Im m \mathcal{H}_{A}$
shown in Figs.
\ref{due} and \ref{uno} respectively \cite{Guidal:2013rya}:
\begin{equation}
\Re e \mathcal{H}_{A} (\xi,t) \equiv 
{\cal P} \int_0^1 dx H_+(x,\xi,t) C_+(x,\xi) \, ,
\label{ReC}
\end{equation}
and
\begin{equation}
  \Im m \mathcal{H}_{A} = H_+(\xi,\xi,t)  \, ,
\end{equation}
with:
\begin{equation}
    H_+ = H(x,\xi,t)-H(x,-\xi,t) \, ,
\end{equation}
amd
\begin{equation}
    C_+(x,\xi) = \frac{1}{x+\xi}+\frac{1}{x-\xi} \, .
\end{equation}


Besides, it is also known that $\Re e(\mathcal{H}_{A})$
satisfies a once subtracted dispersion relation at fixed $t$ and can be therefore related
to $\Im m \mathcal{H}_{A}$, leading to
\cite{Anikin:2007yh,Diehl:2007jb,
Radyushkin:2011dh,
Pasquini:2014vua}
\begin{equation}
\Re e \mathcal{H}_{A} (\xi,t) \equiv
{\cal P} \int_0^1 dx H_+(x,x,t) C_+(x,\xi)
- \Delta(t)
\label{disp}
\end{equation}

One notices that, in
contrast to the convolution integral entering the real part of the CFF in 
\eqref{ReC}, where
the GPD enters for unequal values of its first and second argument, the integrand in the
DR (spectral function) corresponds to the GPD where its first and second arguments
are equal.
The subtraction term $\Delta(t)$ can be related to the
so called $d-$term and
accurate measurements and precise calculations would allow to access therefore
the nuclear $d-$term.
This quantity, introduced initially to recover polinomiality
in DDs approaches to GPDs modelling \cite{Polyakov:1999gs},
can be related to the form factor of the QCD EMT
(see e.g. Ref. \cite{Polyakov:2018zvc}).
It encodes information on the distribution of forces and pressure between elementary QCD degrees of freedom
in the nucleus. For
nuclei, it has been predicted to behave as $A^{7/3}$ in a mean field scheme, either in the liquid drop model of nuclear structure \cite{Polyakov:2002yz}
or in the Walecka model \cite{Jung:2014jja}.
None of these approaches makes much sense for light nuclei.
Accurate realistic calculations are possible in the latter case.
Using light nuclei
one would therefore explore, at the parton level, the
onset and evolution of the mean field
behavior across the periodic table, from deuteron to finite nuclei.

\vskip 0.5cm

In this sense, the $^3$He target acquires an important role: 
an intermediate behavior is expected between that of the almost unbound
deuteron system and that of the deeply bound alpha particle.
The formalism would follow that already presented for the proton, 
a spin one-half target, in terms of CFFs defining proper spin dependent asymmetries.
Realistic theoretical calculations
are available for GPDs\cite{Scopetta:2004kj,Scopetta:2009sn,Rinaldi:2012ft,Rinaldi:2012pj} and are in progress for the relevanto CFFS, cross sections and asymmetries. 

\vskip 0.5cm

Needless to say, incoherent DVCS off He nuclei at JLab 12, in particular tagged using the detector
developed by the ALERT run group \cite{Armstrong:2017zcm},
performed with electron and positron beams,
would allow the measurement of the $D-$term for the bound nucleon, either proton (tagging 2H from DVCS on $^3$He
or $^3$H from DVCS on $^4$He) or neutron (tagging $^3$He from DVCS on $^4$He).
Modifications of the $D-$term of the nucleon in the nuclear medium, studied e.g. in \cite{Jung:2014jja},
would be at hand, as well as a glimpse at the transverse structure of the neutron, complementarytothat obtained with deuteron targets.

\vskip 0.5cm

We note on passing that, in principle, from the measurement of $\tilde \sigma_{DVCS}$ using electron and positron beams in coherent DVCS on $^4$He,
for the first time higher twist CFFs would be studied for a spinless target...
To be developed????????????






%\section*{Results}

%Just for kicks here's a citation \cite{Gustafsson2016}. And a %reference to a supplement \cref{note:Note1}. And %\nameref{note:Note1}.
%\Blindtext

\begin{figure}%[tbhp]
%\centering
\includegraphics[width=.7\linewidth, angle=270]{Figures/imxb.eps}
\caption{The imaginary part of the CFF measured in coherent DVCS off $^4$He.
Data from Ref. \cite{Hattawy:2017woc}; calculations (red crosses) from
Ref. \cite{Fucini:2018gso}}
\label{uno}
%\end{figure}
%\begin{figure}%[tbhp]
%\centering
\includegraphics[width=.7\linewidth, angle=270]{Figures/rexb.eps}
\caption{
The imaginary part of the CFF measured in coherent DVCS off $^4$He.
Data from Ref. \cite{Hattawy:2017woc}; calculations (red crosses) from
Ref. \cite{Fucini:2018gso}
}
\label{due}
\end{figure}



%\Blindtext

%Figure \ref{fig:computerNo} shows an example of how to insert a column-wide figure. To insert a figure wider than one column, please use the \verb|\begin{figure*}...\end{figure*}| environment. Figures wider than one column should be sized to 11.4 cm or 17.8 cm wide. Use \verb|\begin{SCfigure*}...\end{SCfigure*}| for a wide figure with side captions.

%\Blindtext \Blindtext \Blindtext

%\section*{Conclusions}

%$blablaba \ref{fig:computerNo} 
%\blindtext

%\subsection*{Blabla} 
%\blindtext

%\section*{Conclusions}

%blablaba \ref{fig:computerNo}
%\blindtext

%\begin{acknowledgements}
%\blindtext
%\end{acknowledgements}

%\newpage

\section*{Bibliography}
\bibliography{MyWPCont}

%% You can use these special %TC: tags to ignore certain parts of the text.
%TC:ignore
%the command above ignores this section for word count
\onecolumn
\newpage

%\section*{Word Counts}
%This section is \textit{not} included in the word count. 
%\subsection*{Notes on Nature Methods Brief Communication}
%\begin{itemize}
%\item Abstract: 3 sentences, 70 words.
%\item Main text: 3 pages, 2 figures, 1000-1500 words, more figures possible if under 3 pages
%\end{itemize}

%\subsection*{Statistics on word count}
%\detailtexcount
%\newpage

%%%%%%%%%%%%%%%%%%%%%%%%%%%%%
% Supplementary Information %
%%%%%%%%%%%%%%%%%%%%%%%%%%%%%
\captionsetup*{format=largeformat}
\section{Full list of authors and affiliations} \label{note:Note1} 

S.~Fucini$^a$, M.~Hattawy$^b$, M.~Rinaldi$^a$, S.~Scopetta$^a$  

\begin{center}

{\it

$^a${Dipartimento di Fisica e Geologia, Università degli studi di Perugia, and INFN, sezione di Perugia, \\
via A. Pascoli snc, 06123, Perugia, Italy}

$^b${Old Dominion VA USA ????}
}
\end{center}

%TC:endignore
%the command above ignores this section for word count

\end{document}
